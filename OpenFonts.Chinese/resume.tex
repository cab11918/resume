%%%%%%%%%%%%%%%%%%%%%%%%%%%%%%%%%%%%%%%
% Deedy - One Page Two Column Resume
% LaTeX Template
% Version 1.2 (16/9/2014)
%
% Original author:
% Debarghya Das (http://debarghyadas.com)
%
% Original repository:
% https://github.com/deedydas/Deedy-Resume
%
% IMPORTANT: THIS TEMPLATE NEEDS TO BE COMPILED WITH XeLaTeX
%
% This template uses several fonts not included with Windows/Linux by
% default. If you get compilation errors saying a font is missing, find the line
% on which the font is used and either change it to a font included with your
% operating system or comment the line out to use the default font.
% 
%%%%%%%%%%%%%%%%%%%%%%%%%%%%%%%%%%%%%%
% 
% TODO:
% 1. Integrate biber/bibtex for article citation under publications.
% 2. Figure out a smoother way for the document to flow onto the next page.
% 3. Add styling information for a "Projects/Hacks" section.
% 4. Add location/address information
% 5. Merge OpenFont and MacFonts as a single sty with options.
% 
%%%%%%%%%%%%%%%%%%%%%%%%%%%%%%%%%%%%%%
%
% CHANGELOG:
% v1.1:
% 1. Fixed several compilation bugs with \renewcommand
% 2. Got Open-source fonts (Windows/Linux support)
% 3. Added Last Updated
% 4. Move Title styling into .sty
% 5. Commented .sty file.
%
%%%%%%%%%%%%%%%%%%%%%%%%%%%%%%%%%%%%%%%
%
% Known Issues:
% 1. Overflows onto second page if any column's contents are more than the
% vertical limit
% 2. Hacky space on the first bullet point on the second column.
%
%%%%%%%%%%%%%%%%%%%%%%%%%%%%%%%%%%%%%%


\documentclass[]{deedy-resume-openfont}
\usepackage{fancyhdr}
    
\pagestyle{fancy}
\fancyhf{}
\linespread{1.2}
\begin{document}

%%%%%%%%%%%%%%%%%%%%%%%%%%%%%%%%%%%%%%
%
%     LAST UPDATED DATE
%
%%%%%%%%%%%%%%%%%%%%%%%%%%%%%%%%%%%%%%
% \lastupdated

%%%%%%%%%%%%%%%%%%%%%%%%%%%%%%%%%%%%%%
%
%     TITLE NAME
%
%%%%%%%%%%%%%%%%%%%%%%%%%%%%%%%%%%%%%%
\namesection{余}{名皓}{ \urlstyle{same}\href{mailto:906924612@qq.com}{906924612@qq.com} | 1738 6042 684
}

%%%%%%%%%%%%%%%%%%%%%%%%%%%%%%%%%%%%%%
%
%     COLUMN ONE
%
%%%%%%%%%%%%%%%%%%%%%%%%%%%%%%%%%%%%%%

\begin{minipage}[t]{0.25\textwidth} 

%%%%%%%%%%%%%%%%%%%%%%%%%%%%%%%%%%%%%%
%     EDUCATION
%%%%%%%%%%%%%%%%%%%%%%%%%%%%%%%%%%%%%%

\section{教育经历} 
\sectionsep

\subsection{悉尼大学}
\descript{硕士学位,主修软件工程}
\location{2022.08-2024.02}
\sectionsep

\subsection{美国东北大学}
\descript{本科学位,计算机科学}
\descript{辅修平面设计}
\location{2017.09-2021.05}
\sectionsep

%%%%%%%%%%%%%%%%%%%%%%%%%%%%%%%%%%%%%%
%     LINKS
%%%%%%%%%%%%%%%%%%%%%%%%%%%%%%%%%%%%%%

\section{链接}
\sectionsep
Blog://  \href{https://cab11918.github.io/hugo-pweb}{\bf 个人网站} \\
% (总计 3 万访客,8 万阅读量) \\    
Github:// \href{https://github.com/cab11918}{\bf cab11918} \\
% (380+ 关注者) \\
% LinkedIn://  \href{https://www.linkedin.com/in/gaocegege}{\bf gaocegege} \\

%%%%%%%%%%%%%%%%%%%%%%%%%%%%%%%%%%%%%%
%     COURSEWORK
%%%%%%%%%%%%%%%%%%%%%%%%%%%%%%%%%%%%%%

% \section{修读课程}
% \subsection{Graduate}
% Advanced Machine Learning \\
% Open Source Software Engineering \\
% Advanced Interactive Graphics \\
% Compilers + Practicum \\
% Cloud Computing \\
% Evolutionary Computation \\
% Defending Computer Networks \\
% Machine Learning \\
% \sectionsep

%%%%%%%%%%%%%%%%%%%%%%%%%%%%%%%%%%%%%%
%     SKILLS
%%%%%%%%%%%%%%%%%%%%%%%%%%%%%%%%%%%%%%

\section{技能}
\sectionsep

\location{Web相关}
HTML \textbullet{}JS/TS \textbullet{} CSS \textbullet{} React \textbullet{} Redux \textbullet{}  \\
网络知识 \\
\sectionsep
% \location{安卓开发}
% Android APIs \textbullet{} Gradle \textbullet{} Firebase \\
% \sectionsep
\location{编程语言}
JS \textbullet{} Java \textbullet{} Python \\
\sectionsep
\location{云计算}
分布式原理 \textbullet{} AWS \\ 腾讯云 \textbullet{} 谷歌云 \textbullet  \\ Docker

\sectionsep

%%%%%%%%%%%%%%%%%%%%%%%%%%%%%%%%%%%%%%
%
%     COLUMN TWO
%
%%%%%%%%%%%%%%%%%%%%%%%%%%%%%%%%%%%%%%

\end{minipage} 
\hfill
\begin{minipage}[t]{0.73\textwidth} 

%%%%%%%%%%%%%%%%%%%%%%%%%%%%%%%%%%%%%%
%     EXPERIENCE
%%%%%%%%%%%%%%%%%%%%%%%%%%%%%%%%%%%%%%

\section{工作经历}
\sectionsep
\runsubsection{数坤科技}
\descript{前端开发工程师}
\location{2021.09 - 2022.07}
\vspace{\topsep}
\begin{tightemize}
    \item 开发和维护数字健康影像辅助诊断平台。项目使用React函数/类组件、
    Redux、Scss样式、浏览器缓存、及canvas实现业务功能(X光片显示/交互、
    AI图像标注/建议指南),使用Fetch和axios与后端服务通信。
    \item 开发入账管理平台,使用React、Reudx及UI组件库,并涉及列表操作、条件筛选、
    导出/下载以及基于URL的平台跳转功能
    \item 开发健康相关Web小程序,计算用户健康指标并给予建议。涉及移动端UI
    组件库,表格校验、以及插图UI样式
    
\end{tightemize}
\sectionsep

%%%%%%%%%%%%%%%%%%%%%%%%%%%%%%%%%%%%%%
%     RESEARCH
%%%%%%%%%%%%%%%%%%%%%%%%%%%%%%%%%%%%%%

\section{项目}
\sectionsep
\runsubsection{{\bf AI文字生成3D模型Web平台}}
\descript{全栈开发}
\location{2023.11}
\begin{tightemize}
    \item AI应用平台,使用React函数组件及其常用Hooks、Redux开发, 实现了文字生成3D模型,查看/保存、调用API浏览第三方作品等功能
    \item 谷歌云部署模型,使用Firebase数据库与Node.js,后端AI生成任务由Python Flask/Celery实现(任务队列、模型脚本调用、进度查询)

    \end{tightemize}

\sectionsep

\runsubsection{{\bf 安卓手机应用 · TravelBud}}
\descript{手机应用开发}
\location{2022.10}
\begin{tightemize}
    \item 关于旅行的安卓应用,使用Java、Android组件、谷歌云以及Firebase数据库实现,
    用户可聊天、更新行程,查看地图/添加地点
    \end{tightemize}

\sectionsep

\runsubsection{{\bf 基于图像的年龄预测模型}}
\descript{代码/模型评估}
\location{2023.05}
\begin{tightemize}
    \item 通过Wiki Face 数据集回归训练 ResNet, AlexNet, VGG 卷积神经网络,微调参数并综合对比结果实现的基于图片的年龄推测
    \end{tightemize}

\sectionsep

\runsubsection{{\href{https://github.com/cab11918/CS4500-SP21-Snarl}{\bf 地下城游戏 · Snarl}}}
\descript{全栈开发}
\location{2021.07}
\begin{tightemize}
    \item 地牢多人游戏,基于Python、观察者模式开发,使用TCP协议及socket库实现
    客户端与服务器通信,并多线程分开处理服务器的通信和游戏界面,使用广度
    搜索实现NPC找到并击杀玩家。
    \end{tightemize}

\sectionsep

\runsubsection{{\href{https://github.com/cab11918/wbdv-fa19-game-webapp}{\bf 在线游戏平台 · myGame}}}
\descript{全栈开发}
\location{2020.01}
\begin{tightemize}
    \item 基于React框架以及Material UI库以及MongoDB数据库开发,从第三方API获取
    游戏资源的在线游戏论坛
    \item 用户可以搜索、收藏、观看游戏视频以及浏览信息、注册、评论、与其他用户互动
    \end{tightemize}
    
\sectionsep

\runsubsection{{\href{https://github.com/cab11918/music-webapp}{\bf React 音乐播放器}}}
\descript{前端开发}
\location{2020.04}
\begin{tightemize}
    \item 一款基于React开发的音乐播放器,使用网易云音乐API获取音乐数据
    \item 可实现播放、暂停、音量调节;可随机/顺序/循环播放,以及进行音乐搜索和
    添加至播放列表
    \end{tightemize}




%%%%%%%%%%%%%%%%%%%%%%%%%%%%%%%%%%%%%%
%     PUBLICATIONS
%%%%%%%%%%%%%%%%%%%%%%%%%%%%%%%%%%%%%%

% \section{Publications} 
% \renewcommand\refname{\vskip -1.5cm} % Couldn't get this working from the .cls file
% \bibliographystyle{abbrv}
% \bibliography{publications}
% \nocite{*}

\end{minipage} 
\end{document}  \documentclass[]{article}
